\documentclass[12pt,a4paper]{article}
\usepackage[utf8]{inputenc}
\usepackage[T1]{fontenc}
\usepackage[brazilian]{babel}
\usepackage{geometry}
\usepackage{xcolor}
\usepackage{listings}
\usepackage{fancyhdr}
\usepackage{titlesec}

\geometry{margin=2.5cm}

% Configuração do código
\lstset{
    language=bash,
    basicstyle=\ttfamily\small,
    keywordstyle=\color{blue}\bfseries,
    commentstyle=\color{gray},
    stringstyle=\color{red},
    numbers=left,
    numberstyle=\tiny\color{gray},
    stepnumber=1,
    numbersep=5pt,
    backgroundcolor=\color{gray!10},
    frame=single,
    breaklines=true,
    breakatwhitespace=true,
    tabsize=2,
    showspaces=false,
    showstringspaces=false
}

% Cabeçalho e rodapé
\pagestyle{fancy}
\fancyhf{}
\fancyhead[L]{\textbf{Guia de Comandos Git}}
\fancyhead[R]{\today}
\fancyfoot[C]{\thepage}

% Títulos
\titleformat{\section}{\Large\bfseries\color{blue!70!black}}{\thesection}{1em}{}
\titleformat{\subsection}{\large\bfseries}{\thesubsection}{1em}{}

\begin{document}

\title{\textbf{Guia de Comandos Git}\\\large{Atualizar Repositório no GitHub}}
\author{}
\date{\today}
\maketitle

\section{Comandos Básicos para Atualizar o Repositório}

\subsection{1. Verificar o Status Atual}
\begin{lstlisting}
cd ~/Mestrado/Dissertação_Saulo
git status
\end{lstlisting}
Mostra quais arquivos foram modificados, adicionados ou removidos.

\subsection{2. Adicionar Arquivos Modificados}
\begin{lstlisting}
# Adicionar todos os arquivos modificados
git add .

# Ou adicionar arquivos específicos
git add principal.tex estilo.sty
\end{lstlisting}

\subsection{3. Fazer Commit das Alterações}
\begin{lstlisting}
git commit -m "Descrição das alterações feitas"
\end{lstlisting}

\textbf{Exemplos de mensagens de commit:}
\begin{lstlisting}
git commit -m "Atualização do capítulo de metodologia"
git commit -m "Correção de referências bibliográficas"
git commit -m "Adição de novas figuras na seção de resultados"
\end{lstlisting}

\subsection{4. Enviar para o GitHub}
\begin{lstlisting}
git push
\end{lstlisting}

\section{Fluxo Completo (Comando Único)}

\begin{lstlisting}
cd ~/Mestrado/Dissertação_Saulo
git add .
git commit -m "Sua mensagem descrevendo as alterações"
git push
\end{lstlisting}

\section{Comandos Úteis Adicionais}

\subsection{Ver Histórico de Commits}
\begin{lstlisting}
git log --oneline -10
\end{lstlisting}

\subsection{Ver Diferenças Antes de Commitar}
\begin{lstlisting}
git diff
\end{lstlisting}

\subsection{Desfazer Alterações em um Arquivo (antes de adicionar)}
\begin{lstlisting}
git checkout -- nome_do_arquivo.tex
\end{lstlisting}

\subsection{Verificar se Há Atualizações no GitHub}
\begin{lstlisting}
git fetch
git status
\end{lstlisting}

\subsection{Atualizar Repositório Local com Mudanças do GitHub}
\begin{lstlisting}
git pull
\end{lstlisting}

\section{Exemplo Prático}

Suponha que você editou \texttt{principal.tex} e \texttt{estilo.sty}:

\begin{lstlisting}
cd ~/Mestrado/Dissertação_Saulo

# Ver o que mudou
git status

# Adicionar os arquivos
git add principal.tex estilo.sty

# Fazer commit
git commit -m "Atualização do documento principal e estilo"

# Enviar para o GitHub
git push
\end{lstlisting}

\section{Informações do Repositório}

\begin{itemize}
    \item \textbf{URL do Repositório:} \url{https://github.com/SauloAndradePinto/mestrado-dissertacao}
    \item \textbf{Branch Principal:} main
    \item \textbf{Token de Acesso:} Já configurado (não é necessário inserir novamente)
\end{itemize}

\section{Observações Importantes}

\begin{enumerate}
    \item Sempre use \texttt{git status} antes de fazer commit para verificar o que será enviado
    \item Escreva mensagens de commit descritivas e claras
    \item O token de acesso já está configurado, então \texttt{git push} funcionará automaticamente
    \item Se houver conflitos, o Git avisará e você precisará resolvê-los antes de fazer push
\end{enumerate}

\section{Dica: Criar Alias para Facilitar}

Você pode adicionar ao arquivo \texttt{\textasciitilde/.bashrc} ou \texttt{\textasciitilde/.zshrc}:

\begin{lstlisting}
alias git-update='git add . && git commit -m "Update" && git push'
\end{lstlisting}

Depois, recarregue o terminal e use apenas:
\begin{lstlisting}
git-update
\end{lstlisting}

\end{document}

